% !TEX root = ../main.tex

% Indicate the main file. Must go at the beginning of the file.

%----------------------------------------------------------------------------------------
% CHAPTER TEMPLATE
%----------------------------------------------------------------------------------------


\chapter{Einleitung} % Main chapter title
\label{chap1} % Change X to a consecutive number; for referencing this chapter elsewhere, use \ref{ChapterX}
\textcolor{red}{eine bis eineinhalb Seitens
}

Im folgenden Kapitel werden die Ausgangslage, die Aufgabenstellung und das Ziel dieser Projektarbeit erläutert. 

%--------------------------------------------------------------------------------
% SECTION Problemstellung
%--------------------------------------------------------------------------------

\section{Ausgangslage} %eventuell Ausgangslage nennen!
\label{sec11} % Change X to a consecutive number

%% Text hier
\textcolor{red}{Nennt bestehende Arbeiten/Literatur zum Thema (Literaturrecherche)/ Stand der Technik: Bisherige Lösungen des Problems und deren Grenzen(Sie beweisen damit, dass Sie das Fachgebiet kennen und das wesentliche Vorwissen aufgearbeitet haben.)}

Die Schweizerischen Bundesbahnen (SBB) betreiben eine umfangreiche Flotte von Schienenfahrzeugen, die verschiedene Kommunikationssysteme nutzen. Der Multifunction Vehicle Bus (MVB) ist eines dieser Systeme und ermöglicht die Datenübertragung zwischen den elektronischen Komponenten und Systemen eines Zuges.

Trotz der Robustheit des MVB können komplexe Störungen auftreten, die den Betrieb beeinträchtigen. Ein zentraler Grund für die Entwicklung eines MVB-Sniffers ist die Notwendigkeit, diese Störungen präzise zu identifizieren und zu analysieren. Durch die Aufzeichnung der Kommunikationsprotokolle können Probleme wie Kommunikationsausfälle diagnostiziert werden.

Ein weiterer Grund für den Einsatz eines MVB-Sniffers ist die Beobachtung und Überwachung spezifischer MVB-Teilnehmer. Bestimmte Komponenten oder Systeme innerhalb eines Zuges können aufgrund ihrer kritischen Funktion oder ihrer Anfälligkeit für Störungen eine besondere Aufmerksamkeit erfordern. Die detaillierte Analyse des MVB fördert auch ein vertieftes Verständnis des Systems.

%---------------------------------------------------------------------------------
% SECTION Aufgabenstellung
%---------------------------------------------------------------------------------

\section{Aufgabenstellung}
\label{sec12} % Change X to a consecutive number

%% Text hier
\textcolor{red}{Aufabgenstellung aus Complesis übernehmen - oder in den Anhang? 
Formuliert das Ziel der Arbeit (Achtung: Ziel und Aufgabe sind nicht zwingend dasselbe! Bitte sauber trennen.)/ Verweist auf die offizielle Aufgabenstellung der betreuenden Person im Anhang; Spezifiziert die Anforderungen an das Resultat der Arbeit/Übersicht über die Arbeit: stellt die folgenden Teile der Arbeit kurz vor (Das erleichtert die Leserführung und schafft Klarheit.)/ (Angaben zum Zielpublikum: nennt das für die Arbeit vorausgesetzte Wissen)/ (Terminologie: Definiert die in  der Arbeit verwendeten Begriffe) (Nur spezielle Fachbegriffe; man kann in der Regel von einem informierten Zielpublikum ausgehen. Wenn ein Glossar (vgl. 6.2.) erstellt wird, erübrigt sich dieser Abschnitt. )}

Die Entwicklung des MVB-Sniffers umfasst mehrere flexible Schritte. Zunächst sollen die Anforderungen an den Sniffer analysiert und eine Recherche zu bestehenden kommerziellen Lösungen durchgeführt werden. Anschliessend wird ein Grobkonzept erarbeitet und die Hardware-Entwicklung begonnen, wobei zunächst Evaluationsboards verwendet werden sollen. Die Firmware-Entwicklung zur Bitstrom-Decodierung und Tests mit dem MVB-Bus-Simulator sind ebenfalls vorgesehen. Bei schnellem Fortschritt kann das Funktionsmuster zu einem Prototyp weiterentwickelt werden, inklusive der Visualisierung der Daten. Alle Schritte sind zu dokumentieren, und ein wissenschaftlicher Bericht über Theorie, Methodik, Resultate und Diskussion ist zu erstellen. Eine detaillierte Aufgabenstellung ist im Anhang zu finden \textcolor{red}{Referenz zu Aufgabenstellung im Anhang}

%---------------------------------------------------------------------------------
% SECTION "Ziel der Arbeit"
%---------------------------------------------------------------------------------

\section{Ziel der Arbeit}
\label{sec13} % Change X to a consecutive number

%% Text hier
\textcolor{red}{Beschreibung der Hauptziele und zu erwartenden Ergebnisse der Arbeit, Minumum Value Product}

Das Hauptziel dieser Projektarbeit ist die Entwicklung eines MVB-Sniffers für die SBB, der Kommunikationsprotokolle in MVB-Netzwerken aufzeichnen kann. Zunächst wird ein Funktionsmuster auf Basis von Evaluationsboards erstellt und parallel wird eine eigene Elektronikplatine (PCB) entwickelt. 

%---------------------------------------------------------------------------------
% SECTION "Ziel der Arbeit"
%---------------------------------------------------------------------------------

\section{Aufbau der Arbeit}
\label{sec14} % Change X to a consecutive number

%% Text hier
\textcolor{red}{Kurzer Überblick über den Aufbau der Arbeit}

