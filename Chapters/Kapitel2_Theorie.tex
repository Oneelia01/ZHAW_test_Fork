% !TEX root = ../main.tex

% Indicate the main file. Must go at the beginning of the file.

%-------------------------------------------------------------------------------
% CHAPTER TEMPLATE
%-------------------------------------------------------------------------------


\chapter{Theoretische Grundlagen} % Main chapter title
\label{Chapter2TheoretischeGrundlagen} % Change X to a consecutive number; for referencing this chapter elsewhere, use \ref{ChapterX}
Der MVB ist ein in der Bahnindustrie verbreiteter Fahrzeugbus, welcher auf dem Prinzip \textit{single master - multiple slaves} aufbaut. In folgenden Abschnitten werden die Grundlagen des Busses aufgezeigt. Alle Daten und Abbildungen in folgendem Kapitel wurden aus der Norm IEC 61375-3-1 entnommen. \cite{MVB_Norm} 

Die Abschnitte beleuchtet die wesentlichen Aspekte des Physical und Link Layers des Datenbusses. Der Fokus liegt auf der Datenübertragungsgeschwindigkeit, dem Bit-Encoding und den Steuermechanismen wie den Start-Delimitern und dem Master-Slave-Prinzip. Neben den technischen Grundlagen, wie der Manchester-Kodierung und den Non-Data-Symbolen, wird die Struktur von Frames und Telegrammen detailliert beschrieben.

%--------------------------------------------------------------------------------
% SECTION "Linklayer"
%--------------------------------------------------------------------------------

\section{Link Layer}

Dieser Abschnitt erläutert den Link Layer und geht auf das Master- sowie das Slave-Frame und den Aufbau der Check Sequenz ein. Diese Teile werden dann zusammengesetzt und das Master/ Slave Prinzip erklärt. Abgeschlossen wird mit der Perioden Zeiten des Buses.

%-----------------------------------
% SUBSECTION "MVB Bus"
%-----------------------------------

%\textcolor{red}{MVB Aufbau, Master Slave Prinzip, Telegrammme, F-Codes}
\subsection{Master-Frame}
\label{sub:MasterFrame}
Das Master-Frame ist der Anfang eines Telegramms und wird immer vom Busmaster gesendet. Nach dem Startbit folgen 8 Bit, welche dem Master-Frame Delimiter in Abbildung \ref{fig:FrameDelimiterMasterSlave} entsprechen. 

\begin{figure}[H]
    \centering
    \begin{minipage}{0.33 \textwidth}
        \centering
        \begin{enumerate}
            \item Master Start-Delimiter
            \item 16 Bit Frame Data
            \begin{enumerate}
                \item F-Code: 4 Bit
                \item Data: 12 Bit
            \end{enumerate}
            \item 8 Bit check sequenze 
            \item End-Delimiter
        \end{enumerate}
    \end{minipage}
    \hfill
    \begin{minipage}{0.65 \textwidth}
        \includegraphics[width = \textwidth]{Figures/Chap2/Grundlagen/MVB_DOKU/Frames und Telegramme/Fig37_MasterFrameFormat.png}
        \caption{Master-Frame Format}
        \label{fig:MasterFrameFormat}
    \end{minipage}
        
\end{figure}

\subsection{Slave-Frame}
\label{sub:SlaveFrame}
Das Slave-Frame folgt direkt nach dem Master-Frame, sofern der angesprochene Slave eingeschaltet und sende fähig ist. Ansonsten würde der Master nach einer definierten maximalen Wartezeit ein neues Telegramm beginnen. \newline
Im Gegensatz zum Master-Frame hat der Slave-Frame verschiedene Längen, je nach F-Code. Je nach Grösse werden ebenfalls mehrere Check-Sequenzen geschickt. Eine Check-Sequenz kann dabei bis zu 64 Bit verifizieren.

\begin{enumerate}
    \item Slave Start-Delimiter (8 Bit)
    \item 16 - 256 Bit Frame Data (individuell)
    \item Check sequence nach maximal 64 Bit Daten 
    \item End-Delimiter
\end{enumerate}

\begin{figure}[H]
    \centering
    \includegraphics[width = 0.8 \textwidth]{Figures/Chap2/Grundlagen/MVB_DOKU/Frames und Telegramme/Fig38_SlaveFrameFormat.png}
    \caption{Slave-Frame Format}
    \label{fig:SlaveFrameFormat}
\end{figure}

\subsection{Check Sequence}
\label{sub:CheckSequenz}
Die Prüfsequenz wird als zyklische Redundanzprüfung (CRC) für die 16, 32 oder 64 Bit an Daten berechnet, welche gemäss dem generativen Polynom in Gleichung \ref{equ:GenerativesPolynom} berechnet werden soll.

\begin{equation}
    G(x) = x^7 + x^6 + x^5 + x^2 + 1
    \label{equ:GenerativesPolynom}
\end{equation}

\subsection{Master / Slave Prinzip}
\label{sub:MasterSlavePrinzip}
Der MVB wird mit einem Master - Slave Prinzip realisiert. Der Master fordert mit einem definierten Master-Frame (Kapitel \ref{sub:MasterFrame}) die Daten der jeweiligen Slave-Teilnehmer an. Der Slave beantwortet die Anfrage mit einem Slave-Frame (Kapitel \ref{sub:SlaveFrame}. Das Master- und Slave-Frame bilden zusammen ein Telegramm (Siehe Abbildung \ref{fig:Fig39_Telegamm_definition.png}). 

\begin{figure}[H]
    \centering
    \includegraphics[width=0.85\linewidth]{Figures/Chap2/Grundlagen/MVB_DOKU/Frames und Telegramme/Fig39_Telegamm_definition.png}
    \caption{Master- und Slave-Frame bilden ein Telegram}
    \label{fig:Fig39_Telegamm_definition.png}
\end{figure}

In der Norm sind die Telegramme in 16 verschiedene Frame-Codes (F-Codes) unterteilt. Dieser Code wird im Master-Frame geschickt. In Abbildung \ref{fig:FCodeListe} ist eine tabellarische Auflistung zu sehen. 

\begin{figure}[H]
    \centering
    \includegraphics[width=0.9\linewidth]{Figures/Chap2/Grundlagen/MVB_DOKU/Frames und Telegramme/F-Code Liste.png}
    \caption{Auflistung aller möglichen F-Codes}
    \label{fig:FCodeListe}
\end{figure}

F-Codes 5,6,7,10 und 11 sind nicht definiert und werden nicht verarbeitet (ungültig). In der jeweiligen Spalte Master-Frame und Slave-Frame sind die jeweiligen Werte eingetragen, welche im Frame enthalten sind und dessen Grösse. 

\subsection{Perioden auf dem Bus}
\label{sub:PeriodeAufDemBus}

Der Bus ist in 4 verschiedenen Perioden aufgebaut. Die 4 Perioden zusammen werden Basic Period genannt. Die 4 Perioden laufen:

\begin{enumerate}
    \item \textbf{Periodic Phase}: In dieser Phase werden Zyklische Daten übertragen
    \item \textbf{Supervisory Phase}: In dieser Phase können Geräte gescannt werden
    \item \textbf{Event Phase}: In dieser Phase werden Event-Daten (Message Data) übertragen. Wird in der Arbeit nicht weiter verfoglt
    \item \textbf{Guard Phase}: Diese wird sich der Busmaster auf die nächste Periode vorbereiten
\end{enumerate}

In der Norm ist die Basic Period wie in Abbildung \ref{fig:BasicPeriod} illustriert.

\begin{figure}[H]
    \centering
    \includegraphics[width=0.9\linewidth]{Figures/Chap2/Grundlagen/BasicPeriod.png}
    \caption{Basic Periode auf dem MVB Bus}
    \label{fig:BasicPeriod}
\end{figure}

%--------------------------------------------------------------------------------
% SECTION "Physical"
%--------------------------------------------------------------------------------

\section{Physical Layer}

Dieser Abschnitt geht detaillierter auf die Übertragung ein und erklärt, wie die Bit encodiert werden, welche Non-Data-Symbols vorhanden sind und wie damit die Start-Delimiter aufgebaut sind. Zudem wird das Übertragungsmedium und die Geschwindigkeit auf dem Bus gezeigt.

\subsection{Bit-Encoding}
\label{sub:BitEncoding}
Die Frame-Data sollen gemäss folgender \textit{Bit-Encoding} (Abb. \ref{fig:manchester_Bit_Encoding}) kodiert werden.

\begin{itemize}
    \item Eine "'1"' soll kodiert werden als ein \textbf{HIGH} in der ersten und dann ein \textbf{LOW} in der zweiten Hälfte des Bit
    \item Eine "'0"' soll kodiert werden als ein \textbf{LOW} in der ersten und dann ein \textbf{HIGH} in der zweiten Hälfte des Bit
\end{itemize}

Das entspricht dem Manchester-Code.

 
\begin{figure}[H]
    \centering
    \includegraphics[width = 0.7 \textwidth]{Figures/Chap2/Grundlagen/MVB_DOKU/Layer/Bit_Encoding.png}
    \caption{Machester Bit-Encoding}
    \label{fig:manchester_Bit_Encoding}
\end{figure}

\subsection{Non Data Symbols} 
\label{sub:NonDataSymbols}
Der Start-Delimiter enthält Non Data Symbols, welche zur Synchronisierung verwendet werden. Diese Symbols werden ebenfalls Manchastercode violations genannt. Folgende Liste zeigt, wie die Symbols (Abb. \ref{fig:NonDataSymbolsEncoding}) kodiert sind.

\begin{itemize}
    \item "'NH"' soll kodiert werden als ein \textbf{HIGH} Signal während eines Bit 
    \item "'NL"' soll kodiert werden als ein \textbf{LOW} Signal während eines Bit
\end{itemize}

\begin{figure}[H]
    \centering
    \includegraphics[width = 0.7 \textwidth]{Figures/Chap2/Grundlagen/MVB_DOKU/Layer/Non_Data_Symbol.png}
    \caption{Non Data Symbols encoding}
    \label{fig:NonDataSymbolsEncoding}
\end{figure}

\subsection{Start-Delimiter}
\label{sub:StartDelimiter}
Der Start-Delimiter hat die Funktion, ein Frame eindeutig zu identifizieren. Hierbei gibt es zwei unterscheidbare Start-Delimiter, der Master-Frame Delimiter und der Slave-Frame Delimiter. In Abbildung \ref{fig:FrameDelimiterMasterSlave} sind beide Start-Delimiter aufgezeigt. Diese Abbildung stammt aus der MVB Case Study der EPFL und ist im Anhang \ref{app:File11} auf Seite 29 zu finden. Gut zu sehen sind die in Kapitel \ref{sub:NonDataSymbols} erwähnten Manchester Violations beim Master-Frame Delimiter an Stelle 1, 2, 4 und 5 und beim Slave-Frame Delimiter an den Stellen 4, 5, 7 und 8. Das \textit{Start Bit} an der Stelle 0 hat die Funktion, den Start nach einer unbestimmten Zeit im Zustand \textit{Idle} zu signalisieren. Dies gehört nicht zum Start-Delimiter (siehe Abbildung \ref{fig:MasterFrameFormat} und \ref{fig:SlaveFrameFormat}). Die Manchester decodierten Start-Delimiter sind somit wie folgt aufgebaut:

\begin{itemize}
    \item Master-Frame Delimiter: [NH, NL, 0, NH, NL, 0, 0, 0]
    \item Slave-Frame Delimiter: [1, 1, 1, NL, NH, 1, NL, NH]
\end{itemize}


\begin{figure}[H]
    \centering
    \includegraphics[width=0.8\linewidth]{Figures/Chap2/Grundlagen/Startdelimiter_Master_Start_Norm.png}
    \caption{Illustration Master- und Slave-Frame Delimiter}
    \label{fig:FrameDelimiterMasterSlave}
\end{figure}

\subsection{Übertragungsmedien}
\label{Übertragungsmedien}

Für den MVB sind in der Norm drei Varianten definiert. Diese sind wie folgt:
\begin{itemize}
    \item \textbf{ESD:} Electrical Short Distance Bus
    Über eine Signalleitung mit Potenzialtrennung mit Optokoppler oder ohne Potenzialtrennung
    \item  \textbf{EMD:} Electrical Middle Distance Bus
    Über verdrillte Signalleitung und Potenzialtrennung über Induktivität
    \item  \textbf{OGF:} Optical Glass Fiber
    Über ein Lichtwellenleiter
\end{itemize}

In Abbildung \ref{fig:TransceiverInterface} ist eine Illustration der drei Übertragungsmedien gezeigt. 

\begin{figure}[H]
    \centering
    \includegraphics[width=0.9\linewidth]{Figures/Chap2/Grundlagen/MVB_DOKU/EMD_ESD_OGF/Fig4_Transceiver interface.png}
    \caption{Transceiver Interfaces}
    \label{fig:TransceiverInterface}
\end{figure}

Die Verbindungen zwischen ESD und EMD erfolgen über einen D-Sub9-Stecker. Tabelle \ref{tab:PinESDEMD} zeigt die Pinbelegung der Stecker sowie deren Funktionen. Dabei wird zwischen \textit{A Bus} und \textit{B Bus} unterschieden, da der MVB redundant geführt ist und auf den jeweiligen Leitungen A und B gespiegelt wird. Die Details zur Redundanz werden nicht weiter behandelt.


\begin{table}[H]
    \centering
    \begin{tabular}{|r||c|c|} \hline
        Pin & ESD & EMD\\ \hline
        1 & A Data Positiv & A Data Positiv\\ \hline
        2 & A Data Negativ & A Data Negativ\\ \hline
        3 & TxE (optional) & TxE (optional)\\ \hline
        4 & B Data Positiv & B Data Positiv\\ \hline
        5 & B Data Negativ & B Data Negativ\\ \hline
        6 & A Bus Ground & A Positiv Pole\\ \hline
        7 & B Bus Ground & A Negativ Pole\\ \hline
        8 & A Bus 5V & B Positiv Pole\\ \hline
        9 & B Bus 5V & B Negativ Pole\\ \hline
    \end{tabular}
    \caption{Pinbelegung für ESD und EMD Stecker}
    \label{tab:PinESDEMD}
\end{table}

Für das Übertragungsmedium OGF ist keine Pinbelegung erforderlich, da zwei separate Lichtwellenleiter verwendet werden: Einer dient zum Senden, der andere zum Empfangen von Signalen.

\subsection{Physical Layer - Geschwindigkeit auf dem Datenbus}
\label{sub:GeschwindigkeitDatenbus}
Die Signalgeschwindigkeit ist in der Norm definiert. Diese lautet 1.5 MBit/s $\pm$ 0.01\% mit Manchester-Encoding (siehe \ref{sub:BitEncoding})

\begin{itemize}
  \item BR (Bitrate): 1.5 MHz oder 1.5 MBit/s
  \item BT (Bittime): 666.7 ns
\end{itemize}