% !TEX root = ../main.tex

%----------------------------------------------------------------------------------------
% German ABSTRACT PAGE
%----------------------------------------------------------------------------------------
%\renewcommand{\extraabstractname}{Abstract} % muss hier bleiben, sonst ist der Titel "Zusammenfassung", weil auf Deutsch gewechselt

\begin{abstract}
%\addchaptertocentry{\abstractname} % Add the abstract to the table of contents
Die Zusammenfassung entspricht einer Miniaturversion des gesamten Dokuments. Gliedere sie ähnlich: Beginne mit dem Kontext und der Motivation für das Projekt, einer kurzen Beschreibung der Methode und der verfügbaren Daten, Ihren Ergebnissen und den Schlussfolgerungen. Beschränke dich auf eine Seite! 

\end{abstract}


%----------------------------------------------------------------------------------------
% ABSTRACT PAGE
%----------------------------------------------------------------------------------------
\renewcommand{\extraabstractname}{Abstract} % muss hier bleiben, sonst ist der Titel "Zusammenfassung", weil auf Deutsch gewechselt
\begin{extraAbstract}
%\addchaptertocentry{\extraabstractname} % Add the abstract to the table of contents

The abstract is like a miniature version of the entire manuscript. Structure it similarly: Begin with the context and motivation for the project, a brief description of the method and available data, your findings, and conclusions. Limit yourself to one page!

\end{extraAbstract}

%----------------------------------------------------------------------------------------
% Vorwort
%----------------------------------------------------------------------------------------
%\renewcommand{\extraabstractname}{Vorwort} 
%\begin{extraAbstract}
%\addchaptertocentry{\extraabstractname} % Add the abstract to the table of contents


%\end{extraAbstract}
